\documentclass[slidestop,compress,brown]{beamer}
%\usepackage[bar]{beamerthemetree}
\mode<presentation>{
%% \usetheme{Madrid}
\usetheme{Boadilla}
\usecolortheme{wolverine}
\setbeamercovered{transparent}
}

\usepackage{fontspec}
\usepackage{polyglossia}
\usepackage{url}
\usepackage{xltxtra}

\pgfdeclareimage[height=1cm]{sanchaya}{sanchaya}
\logo{\pgfuseimage{sanchaya}}

\setmainfont[Script=Kannada]{Navilu}
\setsansfont[Script=Kannada]{Navilu}

\newfontfamily\english{DejaVu Serif}
\newcommand\en[1]{{\english #1}}

\pgfdeclareimage[width=4cm]{who-am-i}{who-am-i}
\pgfdeclareimage[width=4cm]{debian}{debian}
\pgfdeclareimage[width=4cm]{xfce4}{xfce4}

\title[\tiny{\en{FLOSS Development and Contribution}}]{ಮುಕ್ತ ತಂತ್ರಜ್ನಾನ ಅಭಿವೃದ್ಧಿ ಹಾಗೂ ಸಹಕಾರ}
\author[ವಾಸು]{\en{Vasudev Kamath}}
\institute[\en{Debian}]{}
\date[ಹೆಜ್ಜೆ, ಬೆಂಗಳೂರು, ಕರ್ನಾಟಕ]{\en{January 22, 2012}\\
  ಹೆಜ್ಜೆ - ತಂತ್ರಜ್ನಾನ ಮತ್ತು ಕನ್ನಡ\\
  ಬೆಂಗಳೂರು, ಕರ್ನಾಟಕ}

\AtBeginSection[]{
  \begin{frame}
    \frametitle{\en{Outline}}
    \en{\tableofcontents[currentsection]}
  \end{frame}
}

\begin{document}
\begin{frame}
  \titlepage
\end{frame}  

\begin{frame}{\en{Outline}}
  \tableofcontents
\end{frame}

%% Slide 2: Introduction
\section{\en{Introduction}}
\begin{frame}
  \frametitle{ನನ್ನ ಬಗ್ಗೆ ಒಂದಿಷ್ಟು...}
  \begin{columns}[c]
    \column{.6\textwidth}
    \begin{itemize}[<+->]    
    \item<+-| alert@+> ನಾನ್ಯಾರು?...\\
    \item<+-| alert@+> ನಾನೇನು ಮಾಡುತ್ತಿದ್ದೇನೆ?...\\
    \item<+-| alert@+> ಮುಕ್ತ ತಂತ್ರಜ್ನಾನದೊಂದಿಗೆ ನನ್ನ ಒಡನಾಟವೇನು?...\\
    \end{itemize}
    \column{.4\textwidth}    
    \includegraphics[height=5cm,keepaspectratio]{who-am-i}
  \end{columns}
\end{frame}

%% Slide 3: Getting into World of FLOSS
\section{\en{Getting into the World of FLOSS}}
\begin{frame}{ಮುಕ್ತ ತಂತ್ರಜ್ನಾನದ ಜಗತ್ತಿನಲ್ಲಿ ಮೊದಲ ಹೆಜ್ಜೆ - ಹೇಗೆ?}
  \begin{block}{}
    ಮುಕ್ತ ತಂತ್ರಜ್ನಾನದ ಜಗತ್ತಿಗೆ ಕಾಲಿರಿಸಲು ನಾನೇನು ಮಾಡಬೇಕು?...\linebreak[2]
  \end{block}
  \begin{center}
    \begin{tabular}{l c c r}
      \includegraphics[width=2cm,height=2cm,keepaspectratio]{OSI-logo} &
      \includegraphics[width=2cm,height=2cm,keepaspectratio]{gnu} &
      \includegraphics[width=2cm,height=2cm,keepaspectratio]{tux} &
      \includegraphics[width=2cm,height=2cm,keepaspectratio]{beastie}
    \end{tabular}
  \end{center}
  \begin{enumerate}
  \item <+-| alert@+> \en{The Dummy Path}\\
    ಸಾಮಾನ್ಯ ಮಾನವನ ಹಾದಿ\\
  \item<+-| alert@+> \en{The Hacker Path} \\
    ತಂತ್ರಜ್ನಾನ ಪರಿಣತನ ಹಾದಿ\\
  \end{enumerate}
\end{frame}

%% Slide 4 : The Dummy Path
\section{\en{The Dummy Path}}
\begin{frame}{\en{Dummy Path} - ಸಾಮಾನ್ಯ ಮನುಷ್ಯನ ಹಾದಿ}
  \begin{block}{ಸಾಮಾನ್ಯರ ಪ್ರಶ್ನೆ}
    ನಾನು ಒಂದು ಮುಕ್ತ ತಂತ್ರಾಂಶವನ್ನು ಉಪಯೋಗಿಸುತ್ತಿದ್ದೇನೆ. ನಾನು ಅದಕ್ಕೇನಾದರು \en{contribute} ಮಾಡಬೇಕೆಂದಿದ್ದೇನೆ, ಆದರೆ ನಾನು ಪ್ರೊಗ್ರಾಮರ್ ಅಲ್ಲ :(.\linebreak[2]
  \end{block}
  \pause
  \en{\textit{Contribute}} ಮಾಡಲು...\linebreak[2]
  \begin{itemize}[<+->]
  \item<+-| alert@+> ನೀವು ತಂತ್ರಜ್ನರಾಗಿರಬೇಕೆಂದಿಲ್ಲ.
  \item<+-| alert@+> ನಿಮಗೆ ತಂತ್ರಾಂಶದಲ್ಲಿ ಬಳಸಿರುವ \en{programming language} ಬಗ್ಗೆ ನಿಮಗೆ ತಿಳಿದಿರಬೇಕೆಂದಿಲ್ಲ
  \item <+-| alert@+> ತಂತ್ರಾಂಶ ಬಳಸಲು ತಿಳಿದಿದ್ದರೆ ಸಾಕು.
  \end{itemize}  
\end{frame}

%% Slide 5 : Dummy path continue
\begin{frame}{\en{Dummy Path - continued\ldots}}
  \begin{block}<2->
    {ತಂತ್ರಾಂಶವನ್ನು ಬಳಸಿ}
  \end{block}
  \begin{block}<3->
    {ನಿಮ್ಮ ಗೆಳೆಯರಿಗೂ ಬಳಸುವುದನ್ನು ಹಾಗು ನೀವು ಕಲಿತದ್ದನ್ನು ಹೇಳಿ ಕೊಡಿ}
  \end{block}
  \begin{block}<4->
    {\emph{\en{Bug}}ಗಳನ್ನು ರಿಪೋರ್ಟ್ ಮಾಡಿ}
  \end{block}
  \begin{block}<5->
    {\emph{\en{Documentation}} ಬರೆಯಲು ಸಹಕರಿಸಿ}
  \end{block}
  \begin{block}<6->
    {\emph{\en{Documentation}} ಅನ್ನು ಕನ್ನಡಕ್ಕೆ ಅನುವಾದಿಸಲು ಸಹಕರಿಸಿ}
  \end{block}
  \begin{block}<7->
    {ತಂತ್ರಾಂಶವನ್ನು ಕನ್ನಡಕ್ಕೆ ಅನುವಾದಿಸಲು ಸಹಕರಿಸಿ}
  \end{block}
\end{frame}

%% Slide 6 : Help needed projects
\begin{frame}{\en{Dummy Path -} ನೀವೂ ಸಹಕರಿಸಬಹುದು}
  \begin{columns}[c]
    \column{.6\textwidth}
    \begin{block}{\en{KDE Project}}<1->
      \tiny{ಸಂಪರ್ಕಿಸಿ:\\ ಶಂಕರ್ ಪ್ರಸಾದ್ \en{<\texttt{prasad.mvs@gmail.com}>}\\
        \en{Mailing list <\texttt{kde-l10n-kn@lists.kde.org}>}}
    \end{block}
    \begin{block}{\en{Debian Installer and Debconf Templates}}<2->
      \tiny{ಸಂಪರ್ಕಿಸಿ:\\ನಾನು \en{<\texttt{kamathvasudev@gmail.com}>}\\
        \en{Christian Perrier <\texttt{bubulle@debian.org}\en{>}}\\
        \en{Mailing list <\texttt{debian-l10n-kn@lists.debian.org}>}}
    \end{block}
    \begin{block}{\en{Xfce4 Translation Project}}<3->
      \tiny{ಸಂಪರ್ಕಿಸಿ:\\
        ವಾಸುದೇವ ಕಾಮತ್ \en{<\texttt{kamathvasudev@gmail.com}>}\\
        ಶಂಕರ್ ಪ್ರಸಾದ್ \en{<\texttt{prasad.mvs@gmail.com}>}\\
        \en{Mailing list <\texttt{dev@lists.sanchaya.net}>}}
    \end{block}
    \column{.4\textwidth}
    \includegraphics[width=4cm,height=8cm,keepaspectratio]{kde}<1>
    \includegraphics[width=4cm,height=8cm,keepaspectratio]{debian}<2>
    \includegraphics[width=4cm,height=8cm,keepaspectratio]{xfce4}<3>
  \end{columns}
\end{frame}

%% Slide 7: Hacker Path
\section{\en{The Hacker Path}}
\begin{frame}{\en{The Hacker Path -} ಒಂದು \en{example} ಜೊತೆ}
  \begin{block}{ನನ್ನ ಗೆಳೆಯ ಕನ್ನಡ ಡಿಕ್ಷನರಿ ಪಂಡಿತ}
    \texttt{\en{kn.dict.bot@jabber.org}}
  \end{block}
  \begin{itemize}[<+->]
  \item<+-| alert@+> ಏನಿದು?...
  \item<+-| alert@+> \en{Source code} ಎಲ್ಲಿದೆ?... \en{(How to share code and collaborate?)}
    \vfill
    \begin{flushright}
      \begin{block}{\en{Gitorious}}
        \tiny{\en{URL: \url{https://gitorious.org/dictionary-bot}\\
        License: \emph{GPLv3}\\
        Language: \emph{Python}}}
      \end{block}
    \end{flushright}
  \end{itemize}
\end{frame}

%% Slide 8 - About collaboration
\begin{frame}{\en{The Hacker Path - Code Sharing and collaboration}}
  \begin{columns}[c]
    \column{.5\textwidth}<2->
    \begin{block}{\en{GitHub}}
      \url{https://github.com}
     \end{block}
    \begin{block}{\en{Gitorious}}<3->
      \url{https://gitorious.org}
    \end{block}
    \begin{block}{\en{Google Code}}<4->
      \url{http://code.google.com}
    \end{block}
    \begin{block}{\en{Savannah}}<5->
      \url{https://savannah.nongnu.org/}\\
      \url{https://savannah.gnu.org/}
    \end{block}
    \column{.5\textwidth}
    \includegraphics[width=4cm,keepaspectratio]{github}<2>
    \includegraphics[width=4cm,keepaspectratio]{gitorious}<3>
    \includegraphics[width=4cm,keepaspectratio]{code_logo}<4>
    \includegraphics[width=4cm,keepaspectratio]{gnu-monk}<5>
 \end{columns}
\end{frame}

%% Slide 9 - Collaboration tools
\begin{frame}{\en{The Hacker Path - Other Collaboration tools}}
  \begin{itemize}[<+->]
  \item<+-| alert@+> \en{Mailing Lists}
    \begin{itemize}
    \item \en{\texttt{debian-devel@lists.debian.org}}
    \item \en{\texttt{dev@lists.sanchaya.net}}
    \item \en{Google Groups}
    \item \en{Libre List}
    \end{itemize}
  \item<+-| alert@+> \en{Internet Relay Chat}
    \begin{itemize}
    \item \en{\#{}debian-in irc.oftc.net}
    \item \en{\#{}sanchaya irc.freenode.net}
    \item \en{Pidgin}
    \item \en{Chatzilla}
    \end{itemize}
  \item<+-| alert@+> \en{Hacker Meetups}
    \begin{itemize}
      \item \en{DebConf, MiniDebconf's}
      \item \en{FudCon}
      \item \en{Fossmeet \url{http://fossmeet.in}}
      \item ಹೆಜ್ಜೆ!!
    \end{itemize}
  \end{itemize}
\end{frame}

%% Slide 10 - Questions?
\section{\en{Questions?}}
\begin{frame}{ಪ್ರಶ್ನೆಗಳು?\en{\ldots}}
  \begin{center}
    \includegraphics[width=5cm,height=7cm,keepaspectratio]{who-am-i}
  \end{center}
\end{frame}

%% Slide 11 - Thanks
\begin{frame}
  \begin{center}
    \vfill \vfill
    \huge{ಧನ್ಯವಾದಗಳು\linebreak}
  \end{center}
  \vfill
  \tiny{
    \begin{center}
      \begin{tabular}{l c r}
        \multicolumn{2}{l}{\en{Created with:}} \\
        \huge{\en{\XeTeX{}}} &
        \includegraphics[width=2cm,height=2cm,keepaspectratio]{emacs} &
        \includegraphics[width=2cm,height=2cm,keepaspectratio]{debian}
      \end{tabular}
    \end{center}
  }
\end{frame}
\end{document}
